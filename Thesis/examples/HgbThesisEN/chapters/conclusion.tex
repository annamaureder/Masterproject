\chapter{Conclusion}
\label{cha:Conclusion}

In the first project phase, intense research has been conducted in the area of pose estimation in order to detect one main issue to focus on during the master project and thesis. For that, it was quite essential to form an overall perspective of the state-of-the-art regarding pose estimation of articulated objects. During this process, the field of unsupervised pose estimation of unknown objects was frequently referred to as a great potential as the related approaches perform completely independent from user input. By following this direction, the non-rigid registration became a major indicator for possible optimizations. Taking existing, unsupervised methods as reference (see section \ref{unsupervised}), two approaches for the segmentation of an articulated object into its rigid parts were developed. Thereby, the main goal was the reduction of computation steps of the segmentation procedure. The first approach was a linear, straightforward  divide-and-conquer procedure, which recursively divides two 2D point clouds of the same object in different configurations into matching clusters (see chapter \ref{cha:LinearApproach}). The subdividing was realized with a cluster tree and depth-first traversal to segment the point clouds from one side to another. As a next step, all neighboring clusters were verified to be merged, in case of having subdivided a rigid part. After the merging the rigid parts of the objects are obtained. The second approach to be implemented tried to compensate the drawbacks of the linear approach, specifically the segmentation of articulated objects with a skeleton structure like a human (see chapter \ref{cha:FeatureApproach}). It takes advantage of feature matching \cite{FPFH} and RANSAC for the initial alignment of the two point clouds to detect the ``largest rigid part'' - LRP. By taking advantage of the motion constraints of rigid parts linked by joints, linked parts to the LRP could be detected. The reference paper for this approach is represented by Guo et al. \cite{guo2016correspondence}.

\section{Achieved results}
The linear approach aimed for a drastically  decrease of computation steps regarding the segmentation of an articulated object. In order to achieve that, the segmentation only relied on the point coordinates of the input clusters $C_1$ and $C_2$ and the initial orientation of clusters to be sub divided and merged. Regarding simple objects, being composed of a few rigid parts linked like a chain, the approximate rigid parts could be detected. However, the results were not precise regarding points located near an actual joint. In case of more complex objects the approach in its simple form failed. Different optimization possibilities were proposed to counteract most of the emerged issues. Eventually, the knowledge was gained that a more advanced approach would be required for the segmentation of a complex, articulated object. 

As a next step, a feature-based approach has been developed to overcome the main difficulty of detecting reliable point correspondences between $C_1$ and $C_2$. For the initial alignment and the detection of the actual rigid part feature matching in combination with RANSAC was applied. The resulting LRP could be detected successfully with only a few additional, faulty points. This behavior can be explained by the input data in form of a 2D hull of the object which had to be manually created for this specific 2D use case. The contribution of this approach is the following detection of further linked rigid parts to the LRP. It was taken advantage of the constrained translation of rigid parts resulting from their estimated joints. Unlike the approach from Guo et al, only a stepwise rotation around the joint is required to detect a rigid part. For the successful detection of an actual linked rigid part to the LRP joint weights were introduced. Thus, considerably less steps in comparison to the recursive application of the feature matching and RANSAC approach are required. Adjusting the different thresholds used in the implementation, the right number of joints and rigid parts could be detected in two different poses. Furthermore, a successful correspondence of rigid parts and joints could be achieved. For this reason the results are satisfying, if not necessarily precise. 

Overall, the greatest achievement is the extended knowledge about pose estimation of articulated objects taking into account segmentation of and surface registration. As the implementation of basic algorithms required for the two proposed segmentation approaches was done from scratch, a deep understanding of concepts like feature matching could be acquired. Additionally, the  familiarization with scientific papers is definitely noteworthy.

%TODO: major difficulties, problems

\section{Main difficulties and drawbacks}
One main difficulty in the beginning of the research phase posed the great amount of references about the well explored computer vision task of pose estimation. One main issue had to be detected to focus on in the course of the master thesis. In order to familiarize with the topic a general overview hat to be achieved. To gain a deep understanding for the reference approaches the implementation has been conducted in 2D to fully focus on possible segmentation optimizations. By describing a 3D object in 2D important information about its 3D pose get lost. Thus, it is impossible to gain the actual pose without being confronted with ambiguous poses. For this reason for the implementation of the two approaches, no transformation towards the z-axis were assumed. The creation of 2D point clouds of an articulated object in different configurations was another time-consuming procedure, due to the limited number of available specific test data. As the success of the segmentation approaches are directly depend on appropriate import data multiple datasets were manually created. Certain requirements concerning the input data had to be fulfilled such as a certain density of data points to be less error prone for region growing. 

The drawback of the linear approach is that it only achieves successful results if a simple object with a rigid part only linked to maximum two other parts is given as input. For the segmentation of complex objects it generally fails and is therefore not useful for the desired goal.
The main drawback of the proposed feature-based approach is that it is dependent on a successful initial alignment of two different poses. If this first main step fails, the pose of the articulated object can not be successfully extracted. Another drawback is the application of RANSAC as too many iterations are required for a correct alignment of $C_1$ and $C_2$. The main deficit poses the 2D test data which is not ideal for a precise segmentation into rigid parts. However, this condition can be overcome by conducting additional improvements (see section \ref{FutureWork}).
	
\section{Future work}
\label{FutureWork}
To overcome the difficulty of manually creating test data of articulated objects in 2D, the object's hull could be directly computed from the silhouette of the real world object. As a result, a high number of points would available for the segmentation step. 
A next step would be the implementation of the algorithm in 3D based on the implementation of Guo et al. The focus would be thereby to implement an optimized segmentation into rigid parts to reduce the computation steps similar to the approach in 2D. Another interesting outcome would be to take the computed pose of an object as input for a machine learning approach. As a result, a collection of different template shapes could be acquired. With growing learning phase, a considerable accelerated pose estimation mechanism could be provided. Another objective would be to integrate an optimized segmentation approach in an actual pose capture application to transfer the pose of a real object onto a digital character.

