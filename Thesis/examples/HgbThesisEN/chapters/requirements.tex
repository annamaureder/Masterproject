\chapter{Initial position}
\label{cha:requirements}

Taking the existing methods as reference (see chapter \ref{cha:RelatedWork}) a new segmentation approach is developed. Thereby, the main focus is to reduce the computation steps of the correlated correspondence algorithm \cite{CorrelatedCorrespondance} as well as the LRP algorithm \cite {guo2016correspondence}. To fully focus on the segmentation into its rigid part, the 3D reconstruction of the articulated object is assumed to be available.

\section{Goal}

The goal is to segment an articulated mesh $M$ into its unknown number $n$ of rigid parts $\mathcal{P} =  \{P_1,\ldots,P_n\}$ and extract all joints $m$ $\mathcal{J} =  \{J_1,\ldots,J_m\}$ linking those parts in form of a skeleton structure. In general, this is done by non-rigid registration of the point clouds $C_1$ and $C_2$ of an object in two different poses. $C_1$ is thereby used as a \textit{template} to be registered with $C_2$. The main task is to determine a part assignment $P_i$ and the corresponding transformation $T_i$ for all points of the \textit{template} that aligns them with all points of $C_2$. Basically, a divide and conquer approach is implemented to recursively subdivide $C_1$ and $C_2$ into matching sub clusters. 

\section{Assumptions}

The input mesh $M$ is assumed to solely consist of rigid parts that can not be deformed or stretched (e.g. rigid parts of a human) and are linked by joints. Comparing two poses being adopted by the articulated object, the geodesic distance $g(\boldsymbol{p}_i,\boldsymbol{p}_j)$ between two mesh points $\boldsymbol{p}_i(x,y)$ and $\boldsymbol{p}_j(x,y)$ remains constant. Thereby, it is taken advantage of the knowledge that points located on a rigid part $P_i$ have the same transformation $T_i$ . Furthermore, it is assumed that the two poses of $M$ are oriented in the same direction.

\section{Chosen environment}

The initial approach was implemented in Java, using ImageJ as processing library. The chosen environment depends on the following factors:
\begin{itemize}
	\item familiarity and prior experience
	\item complexity
	\item available plug-in for image processing
\end{itemize}
%%
As ImageJ is mainly used for 2D use cases, another implementation would be possible in 3D using PCL in C++. As a result, the attention can be brought to segmentation and visualization in 3D.