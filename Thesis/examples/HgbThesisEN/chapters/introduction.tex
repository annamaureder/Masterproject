\chapter{Introduction}
\label{cha:Introduction}
Pose and motion estimation of articulated objects is a fundamental task in computer vision due to the progressive digitalization of day-to-day processes. A variety of practical applications exist, such as activity recognition, video surveillance and human-computer interfaces. A case of application the thesis emerged from is the pose capture of a real world articulated object used as input for a digital animation process. Thereby, the principal goal is to detect associated rigid parts and joints in two consecutive poses, in order to interpolate the animation between transformed rigid parts.

\section{Problem statement}
Generally, pose estimation of an articulated object can be described as a segmentation problem as the individual rigid parts and joints linking them are desired. A vast majority of existing pose estimation approaches take advantage of assumed object models in 3D and manually labeled joints and rigid parts to determine basic information about an object's pose. Often combined with a machine learning approach the results are promising after a completed training phase. However, unsupervised methods that detect the pose of an unknown object, constitute a great potential as the pose estimation operates completely independent from user input. Among those methods, the non-rigid registration (see section \ref{registration}) is a well-known approach. The input of this algorithm are two or more poses of one articulated object. By merging one \textit{template} pose onto another \textit{query} pose, part correspondences can be determined and the articulated object is segmented into its rigid parts. 
By applying this approach to an unknown input object in two poses, five core questions are formulated to be considered:
%%
\\\begin{itemize}
	\item How and in which form is the input data for pose estimation received?
	\item Which data points correspond to each other in two different poses?
	\item To which rigid part can a data point be assigned to?
	\item How to estimate the joints linking detected rigid parts?
	\item Which joints/rigid parts correspond to each other in two different poses?
\end{itemize}
%%
Many challenges have to be overcome emerging from different stages of the pose estimation procedure (see section \ref{poseEstimation}). To name the most crucial ones, digital input data of an articulated object in the real world has to be captured. Thereby, input noise emerging from low resolution scanning technologies is an essential factor which has to be considered for pose estimation. Furthermore, the ambiguity of body parts poses a significant difficulty especially if the articulated object is composed of symmetric body parts. One of the main drawbacks of current methods is the computational expensive procedure to detect rigid parts. The root of this problem originates from the detection of correct point correspondences between two input meshes. As this directly influences the run time there is a great demand for improvements, especially if real-time applications are required.

\section{Goal}
By means of current approaches in this particular field (see chapter \ref{cha:StateOfTheArt}), my thesis addresses the issue to detect the initial pose of an unknown articulated object given in two poses. The focus lies thereby on reducing the computation steps of the segmentation procedure. Assuming, a digitalized model in form of the object's hull is available, the thesis fully focuses on the segmentation of an articulated object into its rigid parts. The digitalization of the real world object is therefore passed over. The main goal can be formulated as segmenting an articulated object $M$ into its unknown number $n$ of rigid parts $\mathcal{P} =  \{P_1,\ldots,P_n\}$ and extract all joints $m$ $\mathcal{J} =  \{J_1,\ldots,J_m\}$ linking those parts. The input poses are represented by two 2D point clouds $C_1$ and $C_2$ of $M$ in two different poses. $C_1$ is thereby used as a \textit{template} to be registered with $C_2$. The main task is to determine a part assignment $P_i$ and the corresponding transformation $T_i$ for all points of the \textit{template} that aligns them with all points of $C_2$. The main difficulty is that only the unsorted points of $C_1$ and $C_2$ are present. No further information, like manual labeling by the user or an object model as indicator for the number of rigid parts, are available. The only assumption that can be made is that $M$ only consists of rigid parts that can not be deformed or stretched. Comparing two poses being adopted by the articulated object, the geodesic distance $g(\boldsymbol{p}_i,\boldsymbol{p}_j)$ between two mesh points $\boldsymbol{p}_i(x,y)$ and $\boldsymbol{p}_j(x,y)$ remains constant. It is also taken advantage of the knowledge that points located on a rigid part $P_i$ undergo the same transformation $T_i$.

%TODO: method
\section{Methodology}
To accomplish the proposed goal, an analysis of different pose estimation approaches is conducted in chapter \ref{cha:StateOfTheArt}. Additionally, the concept of surface registration is presented (see section \ref{registration}) to provide necessary background knowledge on segmentation. Consequently, two segmentation approaches are implemented taking unsupervised methods (see subsection \ref{unsupervised}) into consideration. Although the referred pose estimation approaches are perform on 3D data sets, it was a conscious decision to conduct the implementation on 2D point clouds. The main advantages include less degrees of freedom of the data points and a possible pose estimation in the absence of 3D reconstructed data. Furthermore, the attention can be brought to an implementation of potential improvements.

The first approach is a straightforward and linear method (see chapter \ref{cha:LinearApproach}) which aims to reduce the computation steps of previous approaches. This is achieved by iteratively subdividing $C_1$ and $C_2$ with a ``divide and conquer'' approach. Corresponding sub clusters are verified to match and in a negative case further subdivided. This attempt does only depend on all point coordinates of $C_1$ and $C_2$ and the orientation of clusters. In case of many linked parts or too dissimilar transformations between $C_1$ and $C_2$ this approach is not reliable as clusters being compared are actually not corresponding to each other. To address the segmentation with focus on articulated objects with a typical skeleton structure (e.g. a human), a feature-based approach (see chapter \ref{cha:FeatureApproach}) is implemented. In this case additional descriptors are extracted for all points of $C_1$ and $C_2$. Those assist on the initial alignment of the input clusters to detect a reliable corresponding rigid part. Proceeding from there, joints can be estimated and all linked rigid parts are detected iteratively. The main reference paper for this approach is from Guo et al \cite{guo2016correspondence}. By demonstrating the outcomes from the proposed approaches (see section \ref{ResultsLRP}) emerged drawbacks, main difficulties and possible potentials are outlined. Furthermore, planned future work is proposed to compensate originated difficulties (see chapter \ref{cha:Conclusion}). All those offer a solid base for further improvements in this area. 




