\chapter{Introduction}
\label{cha:Introduction}

Pose and motion estimation of articulated objects that consist of rigid parts linked by joints is an active field of research due to the growing digitalization of day-to-day processes. A case of application the thesis emerged from is the pose capture an object. Thereby, it is essential to know the corresponding rigid parts and joints in two consecutive frames, in order to interpolate the animation between the moved parts.
%TODO: Write more about pose capture! --> see papers, puppet pose capture, why 2D?

\section{Problem statement}
A vast majority of existing pose estimation approaches take advantage of labeling for joints and the rigid parts to state basic information about the object to be captured. Often combined with a machine learning approach the results are promising with growing learning phase. However, unsupervised methods that detect the pose of an unknown object, constitute a great potential as the detection of a pose operates completely independent from user input. Among those methods, the non-rigid registration (see section \ref{nonrigidregistration}) is a well-known approach. The input of this algorithm are two or more poses of one articulated object. By merging one \textit{template} pose onto another \textit{query} pose, part correspondences can be made and the articulated object is segmented into its rigid parts. 
By applying this approach to an unknown input object in two poses, some core questions need to be considered:
%%
\\\begin{itemize}
	\item How and in which form is the input data received?
	\item Which points correspond to each other in two different poses?
	\item What are the rigid parts of the articulated object?
	\item Where are the joints linking those rigid parts?
	\item Which joints/rigid parts correspond to each other in two different poses?
\end{itemize}
%%
Many challenges have to be overcome emerging from different stages of the pose capture procedure (see section \ref{PoseCapture}). To name the most crucial ones, digital input data of an articulated object in the real worlds has to be captured. Thereby, input noise depending on the resolution and capture form is a important factor for an successful pose estimation. Furthermore, the ambiguity of body parts is one main problem especially if the articulated object has symmetric body parts which is the case for most living beings. One of the main drawbacks of current methods is the computational expensive procedure to detect rigid parts. As this directly influences the run time there is a great demand for improvements in this area, especially if real-time applications are required.

\section{Goal}

By means of current approaches in this particular field (see chapter \ref{cha:StateOfTheArt}), my thesis addresses the issue to detect the initial pose of an unknown articulated object given in two poses. Assuming, a 3D or 2D reconstructed model in form of a point cloud is already available, the thesis fully focuses on the segmentation of an articulated object into its rigid parts. The scanning and reconstruction step is therefore passed over. The main goal is then to segment an articulated object $M$ into its unknown number $n$ of rigid parts $\mathcal{P} =  \{P_1,\ldots,P_n\}$ and extract all joints $m$ $\mathcal{J} =  \{J_1,\ldots,J_m\}$ linking those parts in form of a skeleton structure. The input poses are constituted by two point clouds $C_1$ and $C_2$ of $M$ in two different poses. $C_1$ is thereby used as a \textit{template} to be registered with $C_2$. The main task is to determine a part assignment $P_i$ and the corresponding transformation $T_i$ for all points of the \textit{template} that aligns them with all points of $C_2$. The main difficulty is that only a number of unsorted points of $C_1$ and $C_2$ are present, no further information, like labeling by the user or PCA (principal component analysis) related variables are available. The only assumption that can be made is that $M$ only consists of rigid parts that can not be deformed or stretched (e.g. the rigid parts of a human) and are linked by joints. Comparing two poses being adopted by the articulated object, the geodesic distance $g(\boldsymbol{p}_i,\boldsymbol{p}_j)$ between two mesh points $\boldsymbol{p}_i(x,y)$ and $\boldsymbol{p}_j(x,y)$ remains constant. Thereby, it is taken advantage of the knowledge that points located on a rigid part $P_i$ have the same transformation $T_i$ .

%TODO: method
\section{Methodology}

To answer the questions posed in the beginning, two segmentation approaches are implemented being inspired by State of the Art approaches. The first approach is quite straightforward and linear (see chapter \ref{cha:LinearApproach}) which aims to reduce the computation steps of previous approaches. This is reached by iteratively subdividing $C_1$ and $C_2$ with a ``divide and conquer'' approach. Assumed corresponding sub clusters are verified to match and in this case those clusters are not subdivided any further. This attempt does only depend on the point coordinates of $C_1$ and $C_2$ and the orientation of the clusters being compared. In case of many linked parts or too dissimilar transformations between $C_1$ and $C_2$ this approach is not reliable as clusters being compared do not correspond to each other. To address the segmentation with focus on articulated objects with a typical skeleton structure (e.g. a human), a feature-based approach (\ref{cha:FeatureApproach}) is implemented which offers more information about the points of $C_1$ and $C_2$ apart from their coordinates. Those support for the initial alignment of the input clusters to find a reliable corresponding rigid part. Proceeding from there, joints can be estimated and all linked rigid parts are detected iteratively. The reference paper for this approach is from Mitra et. al \cite{Mitra07}. By showing the outcomes from the two approaches, emerged drawbacks, main difficulties and possible potentials are outlined and discussed. Furthermore, planned future work is proposed to compensate originated difficulties. All those offer a solid base for further improvements in this area. 




