\chapter{Introduction}
\label{cha:Introduction}

%TODO: fluent text without sections

%TODO: problem

\section{Problem}

Pose and motion estimation of objects is an active field of research due to the growing digitalization of day-to-day processes. A vast majority of existing pose estimation methods take advantage of sensors and markers as indicators for joints. Additionally, the rigid parts of an object and its joints are already known. However, unsupervised methods that are completely independent of user input and detect the pose of an unknown object, constitute a great challenge. Among those methods, the non-rigid registration (see section \ref{nonrigidregistration}) is a well-known approach \cite{survey}. The initial project idea was to create a tool for 3D animation purposes using a small puppet to capture its poses in real-time. However, the idea addressed many different challenges, like 3D reconstruction, segmentation, joint and skeleton estimation as well as creating an interface with a 3D animation program. As the implementation of these tasks would go beyond the scope of a thesis project, it was indispensable to break it down into its main areas.  

pose capture --> digitalization, segmentation, correspondence of rigid parts, joints, computational expensive, input of user --> unknown parameters. To take the puppet pose capture as occasion, following considerations have to be done:
%%
\begin{itemize}
	\item How and in which form is the input data (puppet) received?
	\item What are the rigid parts of the puppet?
	\item What are the joints of the puppet?
	\item Which joints/rigid parts correspond to each other in two different poses?	
\end{itemize}
%%

\section{State of the art}
%TODO: state of the art

already existing approaches solve this problem with user input, pose capture contains much more than segmentation --> see chapter state of the art


%TODO: goal + asssumptions + what will not be done
\section{Goal}

goal: having a 2d contour of an unknown articulated object, detect the rigid parts, joints between two different poses. Focus lies on segmentation, data is already available (no scanning/reconstruction). 

To answer those questions extensive research has been conducted on pose estimation to get an idea of the general workflow (see section \ref{PoseCapture}), possible difficulties and potential approaches (see chapter \ref{cha:RelatedWork}). Thereby, the main issue of segmenting an articulated object into its rigid part frequently emerged and for this reason the thesis project focuses on this field.

\section{Main Goal}

The goal is to segment an articulated mesh $M$ into its unknown number $n$ of rigid parts $\mathcal{P} =  \{P_1,\ldots,P_n\}$ and extract all joints $m$ $\mathcal{J} =  \{J_1,\ldots,J_m\}$ linking those parts in form of a skeleton structure. In general, this is done by non-rigid registration of the point clouds $C_1$ and $C_2$ of an object in two different poses. $C_1$ is thereby used as a \textit{template} to be registered with $C_2$. The main task is to determine a part assignment $P_i$ and the corresponding transformation $T_i$ for all points of the \textit{template} that aligns them with all points of $C_2$. Basically, a divide and conquer approach is implemented to recursively subdivide $C_1$ and $C_2$ into matching sub clusters. 

\section{Assumptions}

The input mesh $M$ is assumed to solely consist of rigid parts that can not be deformed or stretched (e.g. rigid parts of a human) and are linked by joints. Comparing two poses being adopted by the articulated object, the geodesic distance $g(\boldsymbol{p}_i,\boldsymbol{p}_j)$ between two mesh points $\boldsymbol{p}_i(x,y)$ and $\boldsymbol{p}_j(x,y)$ remains constant. Thereby, it is taken advantage of the knowledge that points located on a rigid part $P_i$ have the same transformation $T_i$ . Furthermore, it is assumed that the two poses of $M$ are oriented in the same direction.


%TODO: method
\section{Method, research question}

doing experiments with State of the art as help, chapter "Initial" proposes a quite naive linear approach, taking no image features, only divide and conquer . chapter "improvements" focuses on the drawbacks and continues where this simple approach fails --> articulated objects. Starting with an initial alignment to get the largest rigid part of the object --> recursively detect other parts from there. Relates to \cite{Mitra}.

%TODO: roadmap, start position
\section{Roadmap}
Showing the outcomes from the different approaches implemented, the drawbacks are listed and future work is proposed.

\subsection{Prerequisites}
\label{prerequisites}
Assuming, a 3D reconstructed model in form of a point cloud is already available to fully focus on the segmentation of an articulated object in form of a mesh into its rigid parts.


