\chapter{Related Work}
\label{cha:relatedWork}

Here comes the State-of the Art.
An overview of related methods to non-rigid registration for detecting rigid body parts of an articulated object are mentioned by Chang \cite{chang08articulated} and Tam \cite{survey} which mainly use the ICP (iterative closest point) and the PCA (Principal component analysis) to find corresponding body parts. This paper is based on the Correlated Correspondance algorithm \cite{CorrelatedCorrespondance} \cite{Anguelov04} and Symmetrization \cite{Mitra07}. A following work to \cite{chang08articulated} is \cite{chang09range}. Other methods include temporal coherence, markers and user inputs. Another method is from \cite{correspondence}.

\section{Marker, User input}

Here are methods that take advantage of user inputs and markers and are therefore supervised methods. I will focus my work on non-rigid registration methods, which are unsupervised.

\section{Non-rigid Registration}

Here are methods that focus on non-rigid registration to recover the rigid part of an object.

\subsection{EM-algorithm}

\subsection{LRP}

\subsection{Symmetrization}

