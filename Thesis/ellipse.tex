\section{Other approaches}

\subsection{Points-to-Ellipse fitting}

\subsection{Algorithm}

This algorithm only requires one point cloud containing \textit{m} points \{\textit{pt\textsubscript{0}, ..., pt\textsubscript{m}}\}. The basic idea is to segment the non-rigid object  \textit{S\textsubscript{0}} into its rigid parts \textit{part\textsubscript{1}} and {part\textsubscript{2}} by fitting ellipses to its rigid parts. 
\textit{S\textsubscript{0}} is divided perpendicular to its principal axis \textit{p\textsubscript{0}} into two assumed rigid parts \textit{S\textsubscript{left}} and \textit{S\textsubscript{right}}, initially defining the divider \textit{d} with the secondary axis \textit{s\textsubscript{0}}. The points of \textit{S\textsubscript{left}} and \textit{S\textsubscript{right}} are verified to 
form an ellipse by using its formular

\begin{equation}
\dfrac{x^2}{r_1^2} + \dfrac{y^2}{r_2^2} = 1
\end{equation}

Assuming to verify \textit{S\textsubscript{left}} forming an ellipse, \textit{r\textsubscript{1}} is half the length of the principal axis \textit{p\textsubscript{left}} of \textit{S\textsubscript{left}} through its centroid \textit{c\textsubscript{left}}. Furthermore, \textit{r\textsubscript{2}} is half the length of the secondary axis {s\textsubscript{left}} of \textit{S\textsubscript{left}}. Thereby, the centroid \textit{c\textsubscript{left}} needs to be located in the origin (0,0). 
Now, to check whether a point \textit{pt\textsubscript{i}} of \textit{S\textsubscript{left}} is located on the ellipse, the formular is remodeled and its x values is applied. 

\begin{equation}
(1 -  \dfrac{x^2}{r_1^2}) \cdot {r_2^2} = y^2
\end{equation}

The resulting y-value of the ellipse is compared to the points actual y-value. Given a certain threshold $\tau$ a point either accounts to the number of total points lying on the ellipse \textit{n}, or not.

\begin{equation}
n = \sum_{i=0}^{m}\begin{cases}1 \quad if \quad \|pt_i.y^2 - y^2 \| < \tau \\ 0 \quad otherwise\end{cases}
\end{equation}

The algorithm is repeated by sliding \textit{d} in the direction of the highest error \textit{e}. To be continued until the total error \textit{e\textsubscript{total} = e\textsubscript{left} + \textsubscript{right}} reaches its minimum.

\subsection{Steps}

\begin{enumerate}
	\item The centroid \textit{c\textsubscript{0}}  of \textit{S\textsubscript{0}} is computed.
	
	\item The principal axis \textit{p\textsubscript{0}} is computed through \textit{c\textsubscript{0}} and \textit{S\textsubscript{0}} horizontally oriented. 
	
	\item The secondary axis \textit{s\textsubscript{0}}  perpendicular to \textit{p\textsubscript{0}} through \textit{c\textsubscript{0}} is computed.
	
	\item The divider \textit{d} is initialized with the secondary axis \textit{s\textsubscript{0}} to segment \textit{S\textsubscript{0}} into two assumed rigid parts .
	
	\item The points of \textit{S\textsubscript{0}} are either allocated to \textit{S\textsubscript{left}} or \textit{S\textsubscript{right}} depending on its position to \textit{d\textsubscript{0}}.
	
	\item The ellipse formular is applied on \textit{S\textsubscript{left}} and \textit{S\textsubscript{right}}.
	
	\item An error \textit{e\textsubscript{left}} and \textit{e\textsubscript{right}} is obtained implying how many points of \textit{S\textsubscript{left}} and \textit{S\textsubscript{right}} form an ellipse. 
	
	\item The divider \textit{d} is shifted to the direction of the highest error. To be continued from step 5 until the total error \textit{e\textsubscript{total}} doesn't get smaller. 
\end{enumerate}

\subsection{Results}

\subsection{Reusing detected shapes}

After termination of the algorithm, one point cloud can be segmented into its rigid parts P \{part\textsubscript{1}, ..., part\textsubscript{n}\}. Their variables like the ellipses' centroid \textit{c\textsubscript{i}} and radii \textit{r\textsubscript{1}}, \textit{r\textsubscript{2}} can be used to segment similar point clouds in different configurations. As the shapes to be matched are already known, e.g. how they are linked, finding the position to be segmented is a lot easier.