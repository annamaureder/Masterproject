\chapter[Umgang mit Literatur]{Umgang mit Literatur und anderen Quellen}
\label{cha:Literatur}

\paragraph{Anmerkung:}
Der Titel dieses Kapitels ist absichtlich so
lang geraten, dass er nicht mehr in die Kopfzeile der Seiten passt. 
In diesem Fall kann in der \verb!\chapter!-Anweisung 
als optionales Argument \verb![..]! ein verkürzter Text für die
Kopfzeile (und das Inhaltsverzeichnis) angegeben werden:
%
\begin{LaTeXCode}[numbers=none]
\chapter[Umgang mit Literatur]{Umgang mit Literatur und anderen Quellen}
\end{LaTeXCode}

\section{Allgemeines}

Der richtige Umgang mit Quellen ist ein wesentliches Element bei der Erstellung
wissenschaftlicher Arbeiten im Allgemeinen (\sa\ Abschnitt \ref{sec:Plagiarismus}).
Für die Gestaltung von Quellenangaben sind unterschiedlichste Richtlinien in
Gebrauch, bestimmt \ua\ vom jeweiligen Fachgebiet oder Richtlinien von Verlagen und Hochschulen.
Diese Vorlage sieht ein Schema vor, das in den natur\-wissen\-schaftlich-technischen 
Disziplinen üblich ist.%
\footnote{Anpassungen an andere Formen sind relativ leicht möglich.}
Technisch basiert dieser Teil auf \texttt{BibTeX} \cite{Patashnik1988}
\bzw\ \texttt{Biber}%
\footnote{Wird seit Version 2013/02/19 anstelle von \texttt{bibtex} verwendet 
	(s.\ \url{http://biblatex-biber.sourceforge.net/}),}
in Kombination mit dem Paket \texttt{biblatex} \cite{Lehman2016}.


Die Verwaltung von Quellen besteht grundsätzlich aus zwei Elementen: 
\emph{Quellenverweise} im Text beziehen sich auf Einträge im \emph{Quellenverzeichnis}
(oder in mehreren Quellenverzeichnissen).
Das Quellenverzeichnis ist eine
Zusammenstellung aller verwendeten Quellen, typischerweise ganz am Ende des Dokuments.
Wichtig ist, dass jeder Quellenverweis einen zugehörigen, eindeutigen
Eintrag im Quellenverzeichnis aufweist und jedes Element im Quellenverzeichnis auch
im Text referenziert wird.



\section{Quellenverweise}

Um einen Eintrag im Quellenverzeichnis zu erstellen und im Text darauf zu verweisen, stellt \latex ein zentrales Kommando zur Verfügung.

\subsection{Das \texttt{\textbackslash cite} Makro}

Für Quellenverweise im laufenden Text verwendet man die Anweisung
\begin{itemize}
\item[] \verb!\cite{!\textit{Verweise}\verb!}! 
				\quad oder \quad
        \verb!\cite[!\textit{Zusatztext}\verb!]{!\textit{Verweise}\verb!}!.
\end{itemize}

\noindent%
\textit{Verweise} ist eine durch Kommas getrennte Auflistung von $1$--$n$ Quellen-\emph{Schlüsseln}
zur Identifikation der entsprechenden Einträge im Quellenverzeichnis.
Mit \textit{Zusatztext} können Ergänzungstexte zum aktuellen Quellenverweis angegeben
werden, wie \zB Kapitel- oder Seitenangaben bei Büchern.
Einige Beispiele dazu:

\begin{LaTeXCode}[numbers=none]
Mehr dazu findet sich in \cite{Kopka2003}.
\end{LaTeXCode}
$\rightarrow$ Mehr dazu findet sich in \cite{Kopka2003}.

\begin{LaTeXCode}[numbers=none]
Mehr über \emph{Styles} in \cite[Kap.\ 3]{Kopka2003}.
\end{LaTeXCode}
$\rightarrow$ Mehr über \emph{Styles} in \cite[Kap.\ 3]{Kopka2003}.

\begin{LaTeXCode}[numbers=none]
Die Angaben in \cite[S.\ 274--277]{BurgeBurger1999} sind falsch.
\end{LaTeXCode}
$\rightarrow$ Die Angaben in \cite[S.\ 274--277]{BurgeBurger1999} sind falsch.

\begin{LaTeXCode}[numbers=none]
Überholt sind auch \cite{BurgeBurger1999, Patashnik1988, Duden1997}.
\end{LaTeXCode}
$\rightarrow$ Überholt sind auch \cite{BurgeBurger1999, Patashnik1988, Duden1997}.

Die Sortierung der Angaben im letzten Beispiel erfolgt automatisch.



\subsection{Häufige Fehler}

\subsubsection{Verweise außerhalb des Satzes}
Quellenverweise sollten innerhalb oder am Ende eines Satzes (\dah vor
dem Punkt) stehen, nicht \emph{außerhalb}:
%
\begin{center}
\begin{tabular}{rl}
 \textbf{Falsch:}  & \ldots hier ist der Satz aus. \cite{Oetiker2015} Und jetzt geht es weiter \ldots \\
 \textbf{Richtig:} & \ldots hier ist der Satz aus \cite{Oetiker2015}. Und jetzt geht es weiter \ldots
\end{tabular}
\end{center}

\subsubsection{Verweise ohne vorangehendes Leerzeichen}

Ein Quellenverweis ist \emph{immer} durch ein Leerzeichen vom vorangehenden Wort getrennt, niemals wird er (wie etwa eine Fußnote) direkt an das Wort geschrieben:

\begin{center}
\begin{tabular}{rl}
\textbf{Falsch:}  & \ldots hier folgt die Quellenangabe\cite{Oetiker2015} und es geht weiter \ldots \\
\textbf{Richtig:} & \ldots hier folgt die Quellenangabe \cite{Oetiker2015} und es geht weiter \ldots
\end{tabular}
\end{center}

\subsubsection{Zitate}
Falls ein ganzer Absatz (oder mehr) aus einer Quelle zitiert wird,
sollte der Verweis im vorlaufenden Text und nicht
\emph{innerhalb} des Zitats selbst platziert werden. Als Beispiel die folgende Passage
aus \cite{Oetiker2015}:
\begin{quote}
Typographical design is a craft. Unskilled authors often commit
serious formatting errors by assuming that book design is mostly a
question of aesthetics---``If a document looks good artistically,
it is well designed.'' But as a document has to be read and not
hung up in a picture gallery, the readability and
understandability is of much greater importance than the beautiful
look of it.%
\footnote{Man beachte die Verwendung von englischen Hochkommas innerhalb dieses
Zitats.}
\end{quote}
Für das Zitat selbst sollte übrigens die dafür vorgesehene Umgebung
%
\begin{itemize}
 \item[] \verb!\begin{quote}! \emph{Zitierter Text ...} \verb!\end{quote}!
\end{itemize}
%
verwendet werden, die durch beidseitige Einrückungen das
Zitat vom eigenen Text klar abgrenzt und damit die Gefahr von
Unklarheiten (wo ist das Ende des Zitats?) mindert.
Wenn gewünscht, kann das Innere des Zitats auch in Hochkommas verpackt 
\emph{oder} kursiv gesetzt werden -- aber nicht beides!



\subsection{Umgang mit Sekundärquellen}

In seltenen Fällen kommt es vor, dass man eine Quelle \textbf{A} angeben
möchte (oder muss), die man zwar nicht zur Hand -- und damit auch nicht selbst gelesen --
hat, die aber in einer \emph{anderen}, vorliegenden Quelle \textbf{B} zitiert wird.
In diesem Fall wird \textbf{A} als \emph{Original-} oder \emph{Primärquelle} und \textbf{B} 
als \emph{Sekundärquelle} bezeichnet. Dabei sollten folgende Grundregeln beachtet werden:
%
\begin{itemize}
\item
\textbf{Sekundärquellen} nach Möglichkeit überhaupt \textbf{vermeiden}!
\item
Um eine Quelle in der üblichen Form zitieren zu können, muss man sie \textbf{immer selbst
eingesehen} (gelesen) haben!
\item
Nur wenn man die Quelle wirklich \textbf{nicht} beschaffen kann, ist ein Verweis über eine Sekundärquelle
zulässig. In diesem Fall sollten korrekterweise Pri\-mär- und Sekundärquelle \emph{gemeinsam} 
angegeben werden, wie im nachfolgenden Beispiel gezeigt.
\item
\textbf{Wichtig:} Ins Quellenverzeichnis wird \textbf{nur die tatsächlich vorliegende Quelle} 
(\textbf{B}) und nicht die Originalarbeit aufgenommen!
\end{itemize}
%
\textbf{Beispiel:} Angenommen man möchte aus dem berühmten Buch \emph{Dialogo} von Galileo Galilei 
(an das man nur schwer herankommt) eine Stelle zitieren, die man in einem neueren Werk aus dem Jahr 1969 
gefunden hat. Das könnte man \zB\ mit folgender Fußnote bewerkstelligen.%
\footnote{Galileo Galilei, \emph{Dialogo sopra i due massimi sistemi del mondo tolemaico e copernicano}, 
S.~314 (1632). Zitiert nach \cite[S.~59]{Hemleben1969}.} % Alle Seitennummern sind frei erfunden!


\section{Quellenverzeichnis}


Für die Erstellung des Quellenverzeichnisses gibt es in \latex grundsätzlich 
mehrere Möglichkeiten.
Die gängigste Methode ist die Verwendung von \texttt{BibTeX} \cite{Patashnik1988} \bzw 
\texttt{biber}\footnote{\url{http://mirrors.ctan.org/biblio/biber/documentation/biber.pdf}}, 
wie im Folgenden beschrieben.


\subsection{Literaturdaten in BibTeX}
\label{sec:bibtex}

BibTeX ist ein eigenständiges Programm, das aus einer "`Literaturdatenbank"' (eine oder mehrere
Textdateien mit vorgegebener Struktur) ein für \latex geeignetes Quellenverzeichnis
erzeugt. Literatur zur Verwendung von BibTeX findet sich online, \zB \cite{Feder2006, Patashnik1988}.
Die BibTeX-Datei zu dieser Vorlage ist \nolinkurl{references.bib} (im Hauptverzeichnis).

BibTeX-Dateien können natürlich mit einem Texteditor manuell erstellt werden und für
viele Literaturquellen sind bereits fertige BibTeX-Einträge online verfügbar.
Dabei sollte man allerdings vorsichtig sein, denn diese Einträge sind (auch bei großen
Institutionen und Verlagen) \textbf{häufig falsch oder syntaktisch fehlerhaft}!
Man sollte sie daher nicht ungeprüft übernehmen und insbesondere die Endergebnisse genau kontrollieren.
Darüber hinaus gibt es eigene Anwendungen zur Wartung von
BibTeX-Verzeichnissen, wie beispielsweise
\emph{JabRef}.\footnote{\url{http://jabref.sourceforge.net/}}


\subsubsection{Verwendung von \texttt{biblatex} und \texttt{biber}}

Dieses Dokument verwendet \texttt{biblatex} (Version 1.4 oder höher) in Verbindung
mit dem Programm \texttt{biber}, 
das viele Unzulänglichkeiten des traditionellen BibTeX-Work\-flows behebt und dessen Möglichkeiten deutlich erweitert.%
\footnote{Tatsächlich ist \texttt{biblatex} die erste radikale (und längst notwendige) Überarbeitung des mittlerweile stark in die Jahre gekommenen BibTeX-Workflows. Zwar wird dabei BibTex weiterhin für 
die Sortierung der Quellen verwendet, die Formatierung der Einträge und viele andere Elemente werden jedoch ausschließlich über \latex-Makros gesteuert.}
Allerdings sind die in \texttt{biblatex} verwendeten Literaturdaten nicht mehr vollständig 
rückwärts-kompatibel zu BibTeX. Es ist daher in der Regel notwendig, bestehende oder aus
Online-Quellen übernommene BibTeX-Daten manuell zu überarbeiten (\sa\ Abschnitt~\ref{sec:TippsZuBibtex}).

In dieser Vorlage sind die Schnittstellen zu \texttt{biblatex} weitgehend in der Style-Datei 
\nolinkurl{hgbbib.sty} verpackt. Die typische Verwendung in der \latex-Haupt\-datei sieht 
folgendermaßen aus:
%
\begin{LaTeXCode}[numbers=left]
\documentclass[master,german]{hgbthesis}
   ...
\bibliography{references} /+\label{tex:literatur1}+/
   ...
\begin{document}
   ...
\MakeBibliography{Quellenverzeichnis} /+\label{tex:literatur2}+/
\end{document}
\end{LaTeXCode}
%
In der "`Präambel"' (Zeile \ref{tex:literatur1}) wird mit \verb!\bibliography{references}! 
auf eine (modifizierte) BibTex-Datei \nolinkurl{references.bib} verwiesen.%
\footnote{Das Makro 
\texttt{{\bs}bibliography} ist eigentlich ein Relikt aus BibTeX
und wird in \texttt{biblatex} durch die Anweisung \texttt{{\bs}addbibresource} 
ersetzt. Beide Anweisungen sind gleichwertig, allerdings wird oft nur mit 
\texttt{{\bs}bibliography} die zugehörige \texttt{.bib}-Datei im File-Verzeichnis 
der Editor-Umgebung sichtbar.}
Falls mehrere BibTeX-Dateien verwendet werden, können sie in der gleichen Form angegeben werden.

Die Anweisung \verb!\MakeBibliography{..}! am Ende des Dokuments (Zeile~\ref{tex:literatur2})
besorgt die Ausgabe des Quellenverzeichnisses, hier mit dem Titel "`Quellenverzeichnis"'.
Dabei sind zwei Varianten möglich:
%
\begin{description}
\item[\texttt{{\bs}MakeBibliography}] ~ \newline
   Erzeugt ein in mehrere \emph{Kategorien} (s.\ Abschnitt \ref{sec:BibKategorien}) geteiltes Quellenverzeichnis. 
	 Diese Variante wird im vorliegenden Dokument verwendet.
\item[\texttt{{\bs}MakeBibliography[nosplit]}] ~ \newline
   Erzeugt ein traditionelles \emph{einteiliges} Quellenverzeichnis. 
\end{description}


\subsection{Kategorien von Quellenangaben}
\label{sec:BibKategorien}

Für geteilte Quellenverzeichnisse sind in dieser Vorlage folgende Kategorien vorgesehen
(s.\ Tabelle \ref{tab:QuellenUndEintragstypen}):%
\footnote{Diese sind in der Datei \nolinkurl{hgbbib.sty} definiert.
Allfällige Änderungen sowie die Definition zusätzlicher Kategorien sind 
bei Bedarf relativ leicht möglich.}
%
\begin{itemize}
	\item[] \textsf{literature} -- für klassische Publikationen, die gedruckt oder online vorliegen;
	\item[] \textsf{avmedia} -- für Filme, audio-visuelle Medien (auf DVD, CD, \usw);
	\item[] \textsf{software} -- für Softwareprodukte, APIs, Computer Games;
	\item[] \textsf{online} -- für Artefakte, die \emph{ausschließlich} online verfügbar sind.
\end{itemize}
%
Jedes Quellenobjekt wird aufgrund des angegebenen BibTeX-Eintragtyps 
(\texttt{@\emph{type}}) automatisch einer dieser Kategorien 
zugeordnet (s.\ Tabelle~\ref{tab:BibKategorien}).
Angeführt sind hier nur die wichtigsten Eintragstypen, die allerdings die meisten
Fälle in der Praxis abdecken sollten und nachfolgend durch Beispiele erläutert sind.
Alle nicht explizit angegebenen Einträge werden grundsätzlich der Kategorie \textsf{literature} 
zugeordnet.

\begin{table}
\caption{Definierte Kategorien von Quellen und empfohlene BibTeX-Eintragstypen.}
\label{tab:QuellenUndEintragstypen}
\centering
\begin{tabular}{llc}
	\hline
	\emph{Literatur} (\textsf{literature}) & Typ & Seite\\
	\hline
	Buch (Textbuch, Monographie) & \texttt{@book} & \pageref{sec:@book}\\
	Sammelband (Hrsg.\ + mehrere Autoren) & \texttt{@incollection} & \pageref{sec:@incollection} \\
	Konferenz-, Tagungsband & \texttt{@inproceedings} & \pageref{sec:@inproceedings}\\
	Beitrag in Zeitschrift, Journal & \texttt{@article} & \pageref{sec:@article}\\
	Bachelor-, Master-, Diplomarbeit, Dissertation & \texttt{@thesis} & \pageref{sec:@thesis}\\
	Technischer Bericht, Laborbericht & \texttt{@techreport} & \pageref{sec:@techreport}\\
	Handbuch, Produktbeschreibung & \texttt{@manual} & \pageref{sec:@manual}\\
	Norm & \texttt{@standard} & \pageref{sec:@standard}\\
	Gesetzestext, Verordnung etc. & \texttt{@misc} & \pageref{sec:@misc}\\
%
	\hline
	\emph{Audiovisuelle Medien} (\textsf{avmedia}) & \\
	\hline
	Audio (CD) & \texttt{@audio} & \pageref{sec:@audio}\\
	Bild, Foto, Grafik & \texttt{@image} & \pageref{sec:@image}\\
	Video (auf DVD, Blu-ray Disk, online) & \texttt{@video} & \pageref{sec:@video}\\
	Film (Kino) & \texttt{@movie} & \pageref{sec:@movie}\\
%
	\hline
	\emph{Software} (\textsf{software}) & \\
	\hline
	Softwareprodukt oder -projekt & \texttt{@software} & \pageref{sec:@software}\\
	Computer Game & \texttt{@software} & \pageref{sec:@software}\\
%
	\hline
	\emph{Online-Quellen} (\textsf{online}) & \\
	\hline
	Webseite, Wiki-Eintrag, Blog etc. & \texttt{@online} & \pageref{sec:@online-www}
\end{tabular}
\end{table}


\begin{table}
\caption{Kategorien von Quellenangaben und zugehörige BibTeX-Eintragstypen.
Bei geteiltem Quellenverzeichnis werden die Einträge jeder Kategorie in einem
eigenen Abschnitt gesammelt.
Grau gekennzeichnete Elemente sind Synonyme für die jeweils darüber stehenden Typen.}
\label{tab:BibKategorien}
\centering
\definecolor{midgray}{gray}{0.5}
\setlength{\tabcolsep}{4mm}
\begin{tabular}{llll}
	\textsf{literature} & \textsf{avmedia} & \textsf{software} & \textsf{online} \\
	\hline
	\texttt{@book}          & \texttt{@audio}                & \texttt{@software} & \texttt{@online} \\
	\texttt{@incollection}  & \texttt{\color{midgray}@music} & & \texttt{\color{midgray}@electronic} \\
	\texttt{@inproceedings} & \texttt{@video}                & & \texttt{\color{midgray}@www} \\
	\texttt{@article}       & \texttt{@movie}                & &  \\
	\texttt{@thesis}        & \texttt{@software}             & &  \\
	\texttt{@techreport}    &  & &  \\
	\texttt{@manual}        &  & &  \\
	\texttt{@standard}        &  & &  \\
	\texttt{@misc}          &  & &  \\
	\ldots                  &  & &  \\
	\hline
\end{tabular}
\end{table}

%%------------------------------------------------------

\subsection{Gedruckte Quellen (\textsf{literature})}
\label{sec:KategorieLiterature}

Diese Kategorie umfasst alle Werke, die in gedruckter Form publiziert wurden,
also beispielsweise in Büchern, Konferenzbänden, Zeitschriftenartikeln, Diplomarbeiten \usw
In den folgenden Beispielen ist jeweils der BibTeX-Eintrag in der Datei \nolinkurl{references.bib}
angegeben, gefolgt vom zugehörigen Ergebnis im Quellenverzeichnis.


\subsubsection{\texttt{@book}}
\label{sec:@book}
Ein einbändiges Buch (Monographie), das von einem Autor oder mehreren Autoren zur Gänze gemeinsam verfasst und (typischerweise) von einem Verlag herausgegeben wurde.
% 
\begin{itemize}
\item[] 
\begin{GenericCode}[numbers=none]
@book{BurgerBurge2015,
  author={Burger, Wilhelm and Burge, Mark James},
  title={Digitale Bildverarbeitung},
  subtitle={Eine algorithmische Einführung mit Java},
  publisher={Springer-Verlag},
  location={Heidelberg},
  edition={3},
  year={2015},
  hyphenation={german}
}
\end{GenericCode}
\item[\cite{BurgerBurge2015}] \fullcite{BurgerBurge2015}
\end{itemize}
%
\emph{Hinweis:} Die Auflagennummer (\texttt{edition}) wird üblicherweise nur angegeben, 
wenn es \emph{mehr} als eine Ausgabe gibt -- also insbesondere \textbf{nicht für die 1.\ Auflage}, 
wenn diese die einzige ist!
\textbf{ISBN-Nummern} sollte man auch getrost \textbf{weglassen}.

%%------------------------------------------------------

\subsubsection{\texttt{@incollection}}
\label{sec:@incollection}
Ein in sich abgeschlossener und mit einem eigenen Titel versehener
Beitrag eines oder mehrerer Autoren in einem Buch oder Sammelband.
Dabei ist \texttt{title} der Titel des Beitrags, \texttt{booktitle} der Titel des Sammelbands und
\texttt{editor} der Name des Herausgebers.
%
\begin{itemize}
\item[] 
\begin{GenericCode}[numbers=none]
@incollection{BurgeBurger1999,
  author={Burge, Mark and Burger, Wilhelm},
  title={Ear Biometrics},
  booktitle={Biometrics: Personal Identification in Networked Society},
  publisher={Kluwer Academic Publishers},
  year={1999},
  location={Boston},
  editor={Jain, Anil K. and Bolle, Ruud and Pankanti, Sharath},
  chapter={13},
  pages={273-285},
  hyphenation={english}
}
\end{GenericCode}
\item[\cite{BurgeBurger1999}] \fullcite{BurgeBurger1999}
\end{itemize}


%%------------------------------------------------------

\subsubsection{\texttt{@inproceedings}}
\label{sec:@inproceedings}
Konferenzbeitrag, individueller Beitrag in einem Tagungsband.
Man beachte die Verwendung des neuen Felds \texttt{venue}
zur Angabe des Tagungsorts und 
\texttt{location} für den Ort der Publikation (des Verlags).
%
\begin{itemize}
\item[]
\begin{GenericCode}[numbers=none]
@inproceedings{Burger1987,
  author={Burger, Wilhelm and Bhanu, Bir},
  title={Qualitative Motion Understanding},
  booktitle={Proceedings of the Intl.\ Joint Conference on Artificial Intelligence},
  year={1987},
  month={5},
  editor={McDermott, John P.},
  venue={Mailand},
  publisher={Morgan Kaufmann Publishers},
  location={San Francisco},
  pages={819-821},
  hyphenation={english}
}
\end{GenericCode}
\item[\cite{Burger1987}] \fullcite{Burger1987}
\end{itemize}

%%------------------------------------------------------

\subsubsection{\texttt{@article}}
\label{sec:@article}
Beitrag in einer Zeitschrift, einem wissenschaftlichen Journal oder einer Tageszeitung.
Dabei steht \texttt{volume} üblicherweise für den Jahrgang und \texttt{number} für die 
Nummer innerhalb des Jahrgangs. Der Zeitschriftennamen (\texttt{journal} oder
\texttt{journaltitle}) sollte nur in begründeten Fällen abgekürzt werden, um Missverständnisse
zu vermeiden.
%
\begin{itemize}
\item[]
\begin{GenericCode}[numbers=none]
@article{Mermin1989,
  author={Mermin, Nathaniel David},
  title={What's wrong with these equations?},
  journal={Physics Today},
  volume={42},
  number={10},
  year={1989},
  pages={9-11},
  hyphenation={english}
}
\end{GenericCode}
\item[\cite{Mermin1989}] \fullcite{Mermin1989}
\end{itemize}
%
\emph{Hinweis:} Die Angabe einer Ausgabe für \emph{mehrere} Monate ist in \texttt{biblatex} nicht mehr
über das Feld \texttt{month} möglich, denn dieses darf nur mehr \emph{einen} Wert enthalten.
In diesem Fall kann jedoch einfach das \texttt{issue}-Feld verwendet (\zB\ \texttt{issue=\{5/6\}}
im BibTeX-Eintrag zu \cite{Guttman2001}) und das \texttt{number}-Feld weggelassen werden.

%%------------------------------------------------------

\subsubsection{\texttt{@thesis}}
\label{sec:@thesis}
Dieser (neue) Eintragstyp kann allgemein für akademische Abschlussarbeiten verwendet werden. Er ersetzt
insbesondere die bekannten BibTeX-Einträge \texttt{@phdthesis} (für Dissertationen) sowie
\texttt{@mastersthesis} (für Di\-plom- und Masterarbeiten), die allerdings weiterhin verwendet werden können. Zusätzlich ist damit etwa auch die Angabe von Bachelorarbeiten möglich.

\paragraph{Dissertation (Doktorarbeit):}
%
\begin{itemize}
\item[]
\begin{GenericCode}[numbers=none]
@thesis{Eberl1987,
  author={Eberl, Gerhard},
  title={Automatischer Landeanflug durch Rechnersehen},
  type={phdthesis},
  year={1987},
  month={8},
  institution={Universität der Bundeswehr, Fakultät für Raum- und Luftfahrttechnik},
  location={München},
  hyphenation={german}
}
\end{GenericCode}
\item[\cite{Eberl1987}] \fullcite{Eberl1987}
\end{itemize}

\paragraph{Magister- oder Masterarbeit:} ~ \newline
Analog zur Dissertation (s.\ oben), allerdings mit \texttt{type=\{mathesis\}}.%

%%------------------------------------------------------

\paragraph{Diplomarbeit:} ~ \newline
Analog zur Dissertation (s.\ oben), allerdings mit \texttt{type=\obnh\{Diplomarbeit\}}:%
%
\begin{itemize}
\item[]
\begin{GenericCode}[numbers=none]
@thesis{Artner2007,
  author={Artner, Nicole Maria},
  title={Analyse und Reimplementierung des Mean-Shift Tracking-Verfahrens},
  type={Diplomarbeit},
  year={2007},
  month={7},
  institution={University of Applied Sciences Upper Austria, Digitale Medien},
  location={Hagenberg, Austria},
  url={http://theses.fh-hagenberg.at/thesis/Artner07},
  hyphenation={german}
}
\end{GenericCode}
\item[\cite{Artner2007}] \fullcite{Artner2007}
\end{itemize}
%
Der Inhalt des Felds \verb!url={..}! wird dabei automatisch und ohne zusätzliche
Kennzeichnung als URL gesetzt (mit dem \verb!\url{..}! Makro).

%%------------------------------------------------------

\paragraph{Bachelorarbeit:} ~ \newline
Bachelorarbeiten gelten in der Regel zwar nicht als "`richtige"' Publikationen, bei Bedarf müssen sie aber dennoch referenziert werden können. 
%
\begin{itemize}
\item[]
\begin{GenericCode}[numbers=none]
@thesis{Bacher2004,
  author={Bacher, Florian},
  title={Interaktionsmöglichkeiten mit Bildschirmen und großflächigen Projektionen},
  type={Bachelorarbeit},
  year={2004},
  month={6},
  institution={Upper Austria University of Applied Sciences, Medientechnik und {-design}},
  location={Hagenberg, Austria},
  hyphenation={german}
}
\end{GenericCode}
\item[\cite{Bacher2004}] \fullcite{Bacher2004}
\end{itemize}

%%------------------------------------------------------

\subsubsection{\texttt{@techreport}}
\label{sec:@techreport}
Das sind typischerweise nummerierte Berichte (\emph{technical reports}) aus Unternehmen, 
Hochschulinstituten oder Forschungsprojekten.
Wichtig ist, dass die herausgebende Organisationseinheit (Firma, Institut, Fakultät \etc) und 
Adresse angegeben wird. Sinnvollerweise wird auch der zugehörige URL angegeben, sofern vorhanden. 
%
\begin{itemize}
\item[]
\begin{GenericCode}[numbers=none]
@techreport{Drake1948,
  author={Drake, Huber M. and McLaughlin, Milton D. and Goodman, Harold R.},
  title={Results obtained during accelerated transonic tests of the {Bell} {XS-1} airplane in flights to a {MACH} number of 0.92},
  institution={NASA Dryden Flight Research Center},
  year={1948},
  month={1},
  location={Edwards, CA},
  number={NACA-RM-L8A05A},
  url={http://www.nasa.gov/centers/dryden/pdf/...05A.pdf},
  hyphenation={english}
}
\end{GenericCode}
\item[\cite{Drake1948}] \fullcite{Drake1948}
\end{itemize}

%%------------------------------------------------------

\subsubsection{\texttt{@manual}}
\label{sec:@manual}
Dieser Publikationstyp bietet sich jegliche Art von technischer oder anderer Dokumentation an, wie etwa Produktbeschreibungen von Herstellern, Anleitungen, Präsentationen, White Papers \usw Die Dokumentation muss dabei nicht zwingend gedruckt existieren.
%
\begin{itemize}
\item[]
\begin{GenericCode}[numbers=none]
@manual{Mittelbach2016,
  author={Mittelbach, Frank and Schöpf, Rainer and Downes, Michael and Jones, David M. and Carlisle, David},
  title={The \texttt{amsmath} package},
  year={2016},
  month={11},
  version={2.16a},
  url={http://mirrors.ctan.org/macros/latex/required/amsmath/amsmath.pdf},
  hyphenation={english}
}
\end{GenericCode}
\item[\cite{Mittelbach2016}] \fullcite{Mittelbach2016}
\end{itemize}
%
Oft wird bei derartigen Dokumenten kein Autor genannt. Dann wird der Name des \emph{Unternehmens} oder der \emph{Institution} im \texttt{author}-Feld angegeben, allerdings innerhalb einer \textbf{zusätzlichen Klammer} \texttt{\{..\}}, damit das Argument nicht fälschlicherweise als \emph{Vornamen} + \emph{Nachname} interpretiert wird.%
\footnote{Im Unterschied zu BibTeX wird in \texttt{biblatex} bei \texttt{@manual}-Einträgen das Feld \texttt{organization} nicht als Ersatz für \texttt{author} akzeptiert.}
Dieser Trick wird \ua\ im nächsten Beispiel verwendet.


%%------------------------------------------------------

\subsubsection{\texttt{@standard}}
\label{sec:@standard}


Verweise auf Normen (\emph{standards}) werden in \texttt{biblatex} durch den Typ \texttt{@standard}
unterstützt. Hier ein typisches Beispiel:
%
\begin{itemize}
\item[]
\begin{GenericCode}[numbers=none]
@standard{W3C2017HTML52,
  author={{World Wide Web Consortium}},
  title={HTML 5.2},
  titleaddon={W3C Candidate Recommendation},
	date={2017-08-08},
  url={https://www.w3.org/TR/html52/},
	hyphenation={english}
}
\end{GenericCode}
\item[\cite{W3C2017HTML52}] \fullcite{W3C2017HTML52}
\end{itemize}
%



%%------------------------------------------------------

\subsubsection{\texttt{@misc}}
\label{sec:@misc}
Sollte mit den bisher angeführten Eintragungstypen für gedruckte Publikationen
nicht das Auslangen gefunden werden, sollte man sich zunächst die weiteren (hier nicht näher beschriebenen) 
Typen im \texttt{biblatex}-Handbuch \cite{Lehman2016} ansehen, beispielsweise
\texttt{@collection} für einen Sammelband als Ganzes (also nicht nur ein Beitrag darin)
oder \texttt{@patent} für ein Patent oder eine Patentanmeldung.

Wenn nichts davon passt, dann kann auf den Typ \texttt{@misc} zurückgegriffen werden, der ein
Textfeld \texttt{howpublished} vorsieht, in dem die Art der Publikation individuell 
angegeben werden kann. Das folgende Beispiel zeigt die Anwendung für einen Gesetzestext 
(\sa\ \cite{FhStG1993} und \cite{EuRichtlinie2000}).
%
\begin{itemize}
\item[]
\begin{GenericCode}[numbers=none]
@misc{OoeRaumordnungsgesetz1994,
  title={Oberösterreichisches Raumordnungsgesetz 1994},
  howpublished={LGBl 1994/114 idF 1995/93},
  url={http://www.ris.bka.gv.at/Dokumente/LrOO/...538.pdf},
  hyphenation={german}
}
\end{GenericCode}
\item[\cite{OoeRaumordnungsgesetz1994}] \fullcite{OoeRaumordnungsgesetz1994}
\end{itemize}


\subsection{Filme und audio-visuelle Medien (\textsf{avmedia})}
\label{sec:KategorieAvmedia}
Diese Kategorie ist dazu vorgesehen, audio-visuelle Produktionen wie Filme, 
Tonaufzeichnungen, Audio-CDs, DVDs, VHS-Kassetten \usw\ zu erfassen.
Damit gemeint sind Werke, die in physischer (jedoch nicht in gedruckter) Form
veröffentlicht wurden.
Nicht gemeint sind damit audio-visuelle Werke (Tonaufnahmen, Bilder, Videos) 
die ausschließlich online verfügbar sind -- diese sollten mit einem Elementtyp 
\texttt{@online} (s.\ Tabelle~\ref{tab:BibKategorien} und Abschnitt~\ref{sec:KategorieOnline}) ausgezeichnet werden.

Die nachfolgend beschriebenen Typen \texttt{@audio}, \texttt{@video} und \texttt{@movie} 
sind \emph{keine} Bib\-TeX-Standardtypen. Sie sind aber in \texttt{biblatex} vorgesehen
(und implizit durch \texttt{@misc} ersetzt) und werden hier empfohlen, um die automatische 
Gliederung des Quellenverzeichnisses zu ermöglichen.


\subsubsection{\texttt{@audio}}
\label{sec:@audio}
Hier ein Beispiel für die Spezifikation einer Audio-CD:
%
\begin{itemize}
\item[] 
\begin{GenericCode}[numbers=none]
@audio{Zappa1995,
  author={Zappa, Frank},
  title={Freak Out},
  howpublished={Audio-CD},
  year={1995},
  month={5},
  note={Rykodisc, New York},
  hyphenation={english}
}
\end{GenericCode}
\item[\cite{Zappa1995}] \fullcite{Zappa1995}
\end{itemize}
%
Anstelle von \verb!howpublished={Audio-CD}! könnte auch \verb!type={audiocd}! verwendet werden.


\subsubsection{\texttt{@image}}
\label{sec:@image}

Das nachfolgende Beispiel zeigt den Verweis auf ein digital verfügbares Foto,
das auch in Abb.\ \ref{fig:CocaCola} verwendet wird:
%
\begin{itemize}
\item[] 
\begin{GenericCode}[numbers=none]
@image{CocaCola1940,
  author={Wolcott, Marion Post},
	title={Natchez, Miss.},
	note={Library of Congress Prints and Photographs Division Washington, Farm Security Administration/Office of War Information Color Photographs},
	year={1940},
	month={8},
	url={http://www.loc.gov/pictures/item/fsa1992000140/PP/},
  hyphenation={english}
 }
\end{GenericCode}
\item[\cite{CocaCola1940}] \fullcite{CocaCola1940}
\end{itemize}





\subsubsection{\texttt{@video}}
\label{sec:@video}

Das nachfolgende Beispiel zeigt den Verweis auf ein YouTube-Video:
%
\begin{itemize}
\item[]
\begin{GenericCode}[numbers=none]
@video{HistoryOfComputers2008,
  title={History of Computers},
	url={http://www.youtube.com/watch?v=LvKxJ3bQRKE},
  year={2008},
  month={9},
  hyphenation={english}
}
\end{GenericCode}
\item[\cite{HistoryOfComputers2008}] \fullcite{HistoryOfComputers2008}
\end{itemize}

\noindent
Hier ein Beispiel für den Verweis auf eine DVD-Edition:
%
\begin{itemize}
\item[] 
\begin{GenericCode}[numbers=none]
@video{Futurama1999,
  author={Groening, Matt},
  title={Futurama},
  titleaddon={Season 1 Collection},
  howpublished={DVD},
  year={2002},
  month={2},
  note={Twentieth Century Fox Home Entertainment},
  hyphenation={english}
 }
\end{GenericCode}
\item[\cite{Futurama1999}] \fullcite{Futurama1999}
\end{itemize}
%
In diesem Fall ist das angegebene Datum der \emph{Erscheinungstermin}. 
Falls kein eindeutiger Autor namhaft gemacht werden kann, lässt man das
\texttt{author}-Feld weg und verpackt die entsprechenden Angaben im \texttt{note}-Feld, wie im nachfolgenden Beispiel gezeigt.




\subsubsection{\texttt{@movie}}
\label{sec:@movie}
Dieser Eintragstyp ist für Filme reserviert. 
Hier wird von vornherein \emph{kein} Autor angegeben, weil dieser bei 
einer Filmproduktion \ia\ nicht eindeutig zu benennen ist. 
Im folgenden Beispiel (\sa\ \cite{Psycho1960}) sind die betreffenden Daten 
im \texttt{note}-Feld angegeben:%
\footnote{Übrigens achtet \texttt{biblatex} netterweise darauf, dass der  
Punkt am Ende des \texttt{note}-Texts in der Ausgabe nicht verdoppelt wird.}
%
\begin{itemize}
\item[] 
\begin{GenericCode}[numbers=none]
@movie{Nosferatu1922,
  title={Nosferatu -- A Symphony of Horrors},
  howpublished={Film},
  year={1922},
  note={Drehbuch/Regie: F. W. Murnau. Mit Max Schreck, Gustav von Wangenheim, Greta Schröder.},
  hyphenation={english}
}
\end{GenericCode}
\item[\cite{Nosferatu1922}] \fullcite{Nosferatu1922}
\end{itemize}
%
Die Angabe \verb!howpublished={Film}! ist hier sinnvoll, um die Verwechslung
mit einem (möglicherweise gleichnamigen) Buch auszuschließen.



\subsubsection{Zeitangaben zu Musikaufnahmen und Filmen} 

Einen Verweis auf eine bestimmten Stelle in einem Musikstück oder Film kann man 
ähnlich ausführen wie die Seitenangabe in einem Druckwerk.
Besonders legendär (und häufig parodiert) ist beispielsweise die Duschszene
in \emph{Psycho} \cite[T=00:32:10]{Psycho1960}.
Alternativ zur simplen Zeitangabe "`T=\emph{hh}:\emph{mm}:\emph{ss}"' 
könnte man eine bestimmte Stelle auch auf den Frame genau durch 
den zugehörigen \emph{Timecode} "`TC=\emph{hh:mm:ss:ff}"' angeben, 
\zB\ \cite[TC=00:32:10:12]{Psycho1960} für Frame \emph{ff}=12.



\subsection{Software (\textsf{software})}
\label{sec:@software}



Dieser Eintragstyp ist insbesondere für Computerspiele geeignet (in Ermangelung
eines eigenen Eintragstyps).
%
\begin{itemize}
\item[] 
\begin{GenericCode}[numbers=none]
@software{LegendOfZelda1998,
  author={Miyamoto, Shigeru and Aonuma, Eiji and Koizumi, Yoshiaki},
  title={The Legend of Zelda: Ocarina of Time},
  howpublished={N64-Spielmodul},
  publisher={Nintendo},
  year={1998},
  hyphenation={english}
}
\end{GenericCode}
\item[\cite{LegendOfZelda1998}] \fullcite{LegendOfZelda1998}
\end{itemize}

\noindent
Nachfolgend ein Beispiel für den Verweis auf ein typisches Software-Projekt:
%
\begin{itemize}
\item[] 
\begin{GenericCode}[numbers=none]
@software{SpringFramework,
	title={Spring Framework},
	url={https://github.com/spring-projects/spring-framework},
	hyphenation={english}
}
\end{GenericCode}
\item[\cite{SpringFramework}] \fullcite{SpringFramework}
\end{itemize}



\subsection{Online-Quellen (\textsf{online})}
\label{sec:KategorieOnline}

Bei Verweisen auf Online-Resourcen sind grundsätzlich drei Fälle zu unterscheiden:
%
\begin{itemize}
\item[A.] Man möchte allgemein auf eine Webseite verweisen, etwa auf die 
	"`Panasonic products for business"' Seite.%
	\footnote{\url{http://business.panasonic.co.uk/}}
	In diesem Fall wird nicht auf ein konkretes "`Werk"' verwiesen und daher
	erfolgt \emph{keine} Aufnahme ins Quellenverzeichnis. Stattdessen
	genügt eine einfache Fußnote mit \verb!\footnote{\url{..}}!, wie im vorigen
	Satz gezeigt.
\item[B.] Ein gedrucktes oder audio-visuelles Werk 
	(s.\ Abschnitte \ref{sec:KategorieLiterature} und \ref{sec:KategorieAvmedia})
	ist \emph{zusätzlich} auch online verfügbar. In diesem Fall ist die Primär\-publikation 
	aber \emph{nicht} "`online"' und es genügt, ggfs.\ den zugehörigen Link im 
	\texttt{url}-Feld anzugeben, das bei jedem Eintragstyp zulässig ist.
\item[C.] Es handelt sich im weitesten Sinn um ein Werk, das aber 
	\emph{ausschließlich} online verfügbar ist, wie \zB\ ein Wiki oder Blog-Eintrag.
	Die Kategorie \emph{online} ist genau (und \emph{nur}) für diese 
	Art von Quellen vorgesehen.
\end{itemize}


\subsubsection{Beispiel: Wiki-Eintrag}
\label{sec:@online-www}
Durch den Umfang und die steigende Qualität dieser Einträge erscheint
die Aufnahme in das Quellenverzeichnis durchaus berechtigt.
Beispielsweise bezeichnet man als "`Reliquienschrein"'
einen Schrein, in dem die Reliquien eines oder 
mehrerer Heiliger aufbewahrt werden \cite{WikiReliquienschrein2016}.
%
\begin{itemize}
\item[]
\begin{GenericCode}[numbers=none]
@online{WikiReliquienschrein2016,
	title={Reliquienschrein},
	url={https://de.wikipedia.org/wiki/Reliquienschrein},
  year={2016},
  month={8},
  urldate={2017-02-28}
}
\end{GenericCode}
\item[\cite{WikiReliquienschrein2016}] \fullcite{WikiReliquienschrein2016}
\end{itemize}
%
In diesem Fall besteht die Quellenangabe praktisch nur mehr aus dem URL.
Mit \texttt{year} und \texttt{month} kann man die Version näher spezifizieren, 
die zum gegebenen Zeitpunkt aktuell war.
Durch die (optionale) Angabe von \texttt{urldate} (im \texttt{YYYY-MM-DD} Format) wird automatisch 
die Information eingefügt, wann das Online-Dokument tatsächlich eingesehen wurde.




\noindent\textbf{Hinweis:}
Technisch ist bei Online-Quellen nur das Feld \texttt{url} erforderlich,
die Angabe von weiteren Details (\zB\ \texttt{author}) ist aber natürlich möglich.
Liegt aber \emph{kein} Autor vor, dann sollte man -- wie in den obigen Beispielen gezeigt --
zumindest einen sinnvollen \emph{Titel} (\texttt{title}) angeben, 
der für die Sortierung im Quellenverzeichnis verwendet wird.


%\subsubsection{Beispiel: Bildquelle}
%
%In Abbildungen wird sehr häufig fremdes Bildmaterial verwendet, dessen Herkunft natürlich 
%in jedem Fall angegeben werden sollte. Die Angabe von URLs unmittelbar in den Captions von Abbildungen
%ist problematisch, weil sie meistens für das Schriftbild ziemlich störend sind.
%Einfacher ist es, auch einzelne Bilder und Grafiken ins Quellenverzeichnis aufzunehmen und
%wie üblich mit dem \verb!\cite{..}! Kommando darauf zu verweisen.
%Beispiele dazu sind die Angaben zu Abb.\ \ref{fig:CocaCola}--\ref{fig:ibm360}.
%%
%\begin{itemize}
%\item[]
%\begin{GenericCode}[numbers=none]
%@online{IBM360,
	%title={Mainframes},
  %url={http://www.plyojump.com/classes/mainframe_era.php}
%}
%\end{GenericCode}
%\item[\cite{IBM360}] \fullcite{IBM360}
%\end{itemize}

%\subsection{Elektronische Datenträger als Ergänzung zur Arbeit}
%
%Wird der Abschlussarbeit ein elektronischer Datenträger (CD-ROM, DVD, USB-Stick
%etc.) beigelegt, empfiehlt sich die angeführten Webseiten in
%elektronischer Form (vorzugsweise als PDF-Da\-tei\-en) abzulegen
%und die zugehörige Quelle im Quellenverzeichnis mit einem 
%entsprechenden Verweis im \texttt{note}-Feld -- \zB\
%"`Kopie auf USB-Stick (Datei \nolinkurl{xyz.pdf})"' --
%zu versehen.
%Für die Angabe von solchen Dateinamen, die nicht als Online-Link
%zu öffnen sind, ist übrigens die Verwendung von 
%\verb!\nolinkurl{...}! anstelle von \verb!\url{...}! zu empfehlen.


\subsection{Tipps zur Erstellung von BibTeX-Dateien}
\label{sec:TippsZuBibtex}

Die folgenden Dinge sollten bei der Erstellung korrekter BibTeX-Dateien beachtet werden.

\subsubsection{\texttt{month}-Attribut}

Das \texttt{month}-Attribut ist in \texttt{biblatex} (im Unterschied zu BibTeX) numerisch
und wird beispielsweise einfach in der Form \verb!month={8}! (für den Monat August)
angegeben.


\subsubsection{\texttt{hyphenation}-Attribut}

Das \texttt{hyphenation} Attribut ermöglicht den korrekten Satz mehrsprachiger Quellenverzeichnisse. 
Es sollte nach Möglichkeit bei jedem Quelleneintrag angegeben werden, also beispielsweise
\begin{quote}
\verb!hyphenation={german}! \quad oder \quad \verb!hyphenation={english}!
\end{quote}
für eine deutsch- \bzw\ englischsprachige Quelle.


\subsubsection{\texttt{edition}-Attribut}

Mit dem numerischen \texttt{edition}-Feld wird die Auflage eines Werks spezifiziert.
Es ist lediglich die Nummer selbst anzugeben, also etwa
\verb!edition={3}!
bei einer dritten Auflage. Das richtige "`Rundherum"' in der Quellenangabe wird 
in Abhängigkeit von der Spracheinstellung automatisch hinzugefügt 
(\zB\ "`3.\ Auflage"' oder "`3rd edition"').
%
Wie bereits auf Seite \pageref{sec:@book} (unter \texttt{@book}) angemerkt, sollte im Fall einer
\textbf{1.~Auflage} (sofern es keine andere Auflage gibt) das \texttt{edition}-Feld \textbf{nicht} 
angegeben werden!


\subsubsection{Vorsicht bei der Übernahme von fertigen BibTeX-Einträgen}

Viele Verlage und Literatur-Broker bieten fertige BibTeX-Einträge zum Herunterladen an.
Dabei ist jedoch größte Vorsicht geboten, denn diese Einträge sind häufig
unvollständig, inkonsistent oder syntaktisch fehlerhaft!
Sie sollten bei der Übernahme \emph{immer} auf Korrektheit überprüft werden!
Besonders sollte dabei auf die richtige Angabe der Vornamen (VN) und Nachnamen (NN) geachtet werden,
nämlich in der Form%
\footnote{\texttt{and} ist hier ein fixes Trennwort zwischen den Namen der einzelnen Autoren.}
\begin{itemize}
\item[]
\texttt{author=\{\textit{NN1}, \textit{VN1a} \emph{VN1b} and \textit{NN2}, \textit{VN2a} \ldots \}}.
\end{itemize}
Das ist \va\ bei mehrteiligen Nachnamen wichtig, weil sonst
Vor- und Nachnamen nicht korrekt zugeordnet werden können, \zB
\begin{itemize}
\item[]
\texttt{author=\{van Beethoven, Ludwig and ter Linden, Jaap\}}
\end{itemize}
für ein (hypothetisches) Werk der Herren \emph{Ludwig van Beethoven} und \emph{Jaap ter Linden}.
sowie die Angabe von \texttt{volume}, \texttt{number} und \texttt{pages}.
Die Namen von Konferenzen sind sehr oft falsch (auch bei ACM und IEEE).
ISBN-, DOI- und ISSN-Nummern sind entbehrlich und können getrost weggelassen werden.


\subsubsection{Häufige Fehler bei Quellenangaben}

Überprüfen Sie das fertige Quellenverzeichnis sorgfältig auf \emph{Vollständigkeit} und \emph{Konsistenz}. 
Ist bei jeder Quelle klar, wie und wo sie publiziert wurde? 
Sind die Angaben ausreichend, um die Quelle aufzufinden?

Hier ist eine Liste der häufigsten Fehler im Zusammenhang mit dem Quellenverzeichnis:
%
\begin{itemize}
\item
Alle Einträge auf fehlende oder falsch interpretierte Elemente überprüfen!
\item
Alle Namen und Vornamen der Autoren überprüfen, sind die Abkürzungen (der Vornamen) konsistent?
\item
Groß-/Kleinschreibung und Satzzeichen in allen  Einträgen überprüfen und ggfs.\ korrigieren.
\item
Bücher: Verlagsnamen und Verlagsort auf Vollständigkeit, Konsistenz und allfällige
Redundanzen überprüfen.
\item
Alle URLs und DOIs etc.\ \emph{weglassen}, wenn sie nicht unbedingt notwendig sind! Das gilt insbesondere
für Bücher und Konferenzbeiträge.
\item
Journal-Beiträge: Den Namen des Journals immer vollständig ausschreiben, \zB\
"`ACM Transactions on Computer-Human Interaction"' anstelle von 
"`ACM Trans.\ Comput.-Hum.\ Interact."'! Seitenangaben nicht vergessen!
\item
Konferenzbände: 
Tagungsbände einheitlich in der Form "'Proceedings of the \emph{XY Conference on Something} \ldots"'
bezeichnen. Tagungsort angeben, Seitenangaben nicht vergessen!
\item
Bei Techn.\ Berichten, Masterarbeiten und Dissertationen muss die Institution (Department)
angegeben sein!
\end{itemize}
 

\subsubsection{Listing aller Quellen}

Durch die Anweisung \verb!\nocite{*}! -- an beliebiger Stelle im Dokument platziert -- werden \emph{alle} bestehenden Einträge der BibTeX-Datei im Quellenverzeichnis aufgelistet, also auch jene, für die es keine explizite \verb!\cite{}! Anweisung gibt. Das ist ganz nützlich, um während des Schreibens der Arbeit eine aktuelle Übersicht auszugeben. Normalerweise müssen aber alle angeführten Quellen auch im Text referenziert sein!



\section{Plagiat und Paraphrase}
\label{sec:Plagiarismus}

Als \emph{Plagiat} bezeichnet man die Darstellung eines fremden Werks als eigene Schöpfung, 
in Teilen oder als Ganzes, egal ob bewusst oder unbewusst.
Plagiarismus ist kein neues Problem im Hochschulwesen, hat sich aber durch die 
breite Verfügbarkeit elektronischer Quellen in den letzten Jahren dramatisch 
verstärkt und wird keineswegs als Kavaliersdelikt betrachtet.
Viele Hochschulen bedienen sich als Gegenmaßnahme heute ebenfalls elektronischer Hilfsmittel 
(die den Studierenden zum Teil nicht zugänglich sind), und man sollte daher bei jeder 
abgegebenen Arbeit damit rechnen, dass sie routinemäßig auf Plagiatsstellen untersucht wird!
Werden solche erst zu einem späteren Zeitpunkt entdeckt, kann das im schlimmsten Fall sogar 
zur nachträglichen (und endgültigen) Aberkennung des akademischen Grades führen.
Um derartige Probleme zu vermeiden, sollte man eher übervorsichtig agieren und zumindest folgende Regeln beachten:
%
\begin{itemize}
\item
Die Übernahme kurzer Textpassagen ist nur unter korrekter Quellenangabe zulässig, wobei der Umfang (Beginn und Ende) des Textzitats in jedem einzelnen Fall klar erkenntlich gemacht werden muss. 
\item
Insbesondere ist es nicht zulässig, eine Quelle nur eingangs zu erwähnen und nachfolgend wiederholt nicht-ausgezeichnete Textpassagen als eigene Wortschöpfung zu übernehmen. 
\item
Auf gar keinen Fall tolerierbar ist die direkte Übernahme oder \emph{Paraphrase} längerer Textpassagen, egal ob mit oder ohne Quellenangabe. Auch indirekt übernommene oder aus einer anderen Sprache übersetzte Passagen müssen mit entsprechenden Quellenangaben gekennzeichnet sein! 
\end{itemize}
%
Im Zweifelsfall finden sich detailliertere Regeln in jedem guten Buch über wissenschaftliches Arbeiten oder man fragt sicherheitshalber den Betreuer der Arbeit.
